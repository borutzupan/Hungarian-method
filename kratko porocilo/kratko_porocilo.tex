\documentclass[a4paper, 12pt]{article}
\usepackage[slovene]{babel}
\usepackage[utf8]{inputenc}
\usepackage[T1]{fontenc}
\usepackage{lmodern}
\usepackage{amsmath, amsfonts}

\title{Madžarska metoda - kratko poročilo}
\author{Borut Zupan}
\date{5. 4. 2022}

\begin{document}
\maketitle

\section{Madžarksa metoda za dvodelne grafe z utežmi}
\textbf{Podatki:}
\begin{enumerate}
    \item Poln dvodelen graf $G$, kjer je $V(G) = X \cup Y$ dvodelna razdelitev, $X = \{x_1,\ldots,x_n\}$
            in $Y = \{y_1,\ldots,y_m\}$.
    \item Matrika cen povezav $C \in \mathbb{R}^{n\times m}$.
\end{enumerate}
\textbf{Rezultat:}
najcenejše popolno prirejanje v $G$, pri čemer je cena prirejanja $M \subseteq E(G)$ enaka
$$ c(M) = \sum_{x_iy_j \in M}c_{ij}. $$

\textbf{Postopek:}
\begin{enumerate}
    \item Zarotiramo matriko cen povezav tako, da bo stolpcev vsaj toliko kot vrstic in 
            definiramo $k = \min{(n,m)}$. Pojdi na točko 2.
    \item Od elementov vsake vrstice matrike $C$ odštejemo najmanjši element vrstice. Pojdi
            na točko 3.
    \item V tej novi matriki najdi ničlo $N$. Označi jo z zvezdico, če v vrstici in stolpcu ničle
            N ni nobene druge ničle z zvezdico. To nadaljuj za vsak element matrike. Pojdi na točko 4.
    \item Pokrij vsak stolpec, ki vsebuje ničlo z zvezdico. Če je pokritih $k$ stolpcev, pojdi na 
            zadnjo točko. Drugače pojdi na točko 5.
    \item Najdi nepokrito ničlo in jo označi z črtico. Če ni nobene ničle z zvezdico v vrstici te
            ničle s črtico pojdi na točko 6. Drugače, pokrij vrstico z ničlno s črtico in razkrij
            stolpec, ki vsebuje ničlo z zvezdico. Nadaljuj ta postopek, dokler ni več nobene nepokrite
            ničle. Shrani najmanjšo nepokrito vrednost v matriki in pojdi na točko 7.
    \item Skonstruiraj zaporedje ničel s črtico in ničel z zvezdico na naslednji način. Naj bo $N_0$ ničla
            ničla s črtico iz točke 5. Z $N_1$ označimo ničlo z zvezdico, ki je v istem stolpcu kot $N_0$ (če obstaja).
            Z $N_2$ označi ničlo s črtico v isti vrstici kot $N_1$. Nadaljuj dokler se zaporedje ne ustavi
            na ničli s črtico, ki nima nobene ničle z zvedico v njenem stolpcu. V tem zaporedju vse ničle s
            črtico spremeni v ničle z zvedico in vse ničle z zvezdico odznači. Vsa pokritja odpokrij in pojdi
            na točko 4.
    \item Dodaj vrednost iz točke 5 vsakem elementu pokritih vrstic in odštej vrednost iz točke 5 vsakemu
            elementu nepokritih stolpcev. Pojdi nazaj na točko 5.
    \item Najcenejše popolno prirejanje predstavljajo indeski ničel z zvezdico.
\end{enumerate}

\section{Potek in načrt dela}
Projekt bom implementiral v programskem jeziku Python. Do sedaj sem si prebral in razumel
problem, ki ga moram impelemtirati. Na internetu sem pregledal nekaj implementacij tega 
problema. Nekatere delajo z grafi, nekatere z matriko cen. Jaz bom impelementrial metodo,
kjer manipuliraš z matriko cen.
Moj cilj je, da ta mesec implementiram točko $1,2,3,4,7$, težji točki $5,6$ pa bom naredil
na koncu.
Po implementaciji, pa še eksperimentiranje z naključnimi utežmi in preverjanje, če je
algoritem res v najslabšem primeru $O(n^3)$.

\end{document}